%==============================================================================
% Voorbeeld gebruik documentklasse hogent-article
%==============================================================================
%
% Compileren in TeXstudio:
%
% - Zorg dat Biber de bibliografie compileert (en niet Biblatex)
%   Options > Configure > Build > Default Bibliography Tool: "txs:///biber"
% - F5 om te compileren en het resultaat te bekijken.
% - Als de bibliografie niet zichtbaar is, probeer dan F5 - F8 - F5
%   Met F8 compileer je de bibliografie apart.
%
% Als je JabRef gebruikt voor het bijhouden van de bibliografie, zorg dan
% dat je in ``biblatex''-modus opslaat: File > Switch to BibLaTeX mode.

\documentclass{hogent-article}
\usepackage{lipsum} % Voor vultekst
\usepackage[backend=biber]{biblatex}
\usepackage{graphicx}
\usepackage{bmpsize}
\addbibresource{bibliografie.bib}


%------------------------------------------------------------------------------
% Metadata over het artikel
%------------------------------------------------------------------------------

%---------- Titel & auteur ----------------------------------------------------

% TODO: geef werktitel van je eigen voorstel op
\PaperTitle{Agile werken met een fixed-price contracts: is het mogelijk?}
% TODO: geef op welk soort artikel dit is
% Dit is typisch de opdracht en het vak waarvoor dit artikel geschreven is, bv.
% ``Verslag onderzoeksproject Onderzoekstechnieken 2018-2019''
\PaperType{Casus}

% TODO: vul je eigen naam in als auteur, geef ook je emailadres mee!
\Authors{Indy Van Canegem\textsuperscript{1}, Mout Pessemier\textsuperscript{2}, Nante Vermeulen\textsuperscript{3}, Jef Malfliet\textsuperscript{4}} % Authors

% TODO: vul de naam van je co-promotor in.
% Als het hier gaat om een voorstel voor de bachelorproef, dan ben je hier
% verplicht de naam van je co-promotor in te vullen. Zoniet, dan kan je het
% leeg laten.
\CoPromotor{}

% Contactinfo: Geef hier de contactgegevens van elke auteur van het artikel (en
% indien van toepassing ook van de co-promotor).
\affiliation{
    \textsuperscript{1} \href{mailto:nante.vermeulen@student.hogent.be}{nante.vermeulen@student.hogent.be}}
\affiliation{
    \textsuperscript{2} \href{mailto:mout.pessemier@student.hogent.be}{mout.pessemier@student.hogent.be}}
\affiliation{
    \textsuperscript{3} \href{mailto:indy.vancanegem@student.hogent.be}{indy.vancanegem@student.hogent.be}}
\affiliation{
    \textsuperscript{4} \href{mailto:jef.malfliet@student.hogent.be}{jef.malfliet@student.hogent.be}
}

%---------- Abstract ----------------------------------------------------------

\Abstract{Hier schrijf je de samenvatting van je artikel, als een doorlopende tekst van één paragraaf. Wat hier zeker in moet vermeld worden: \textbf{Context} (Waarom is dit werk belangrijk?); \textbf{Nood} (Waarom moet dit onderzocht worden?); \textbf{Taak} (Wat ga je (ongeveer) doen?); \textbf{Object} (Wat staat in dit document geschreven?); \textbf{Resultaat} (Wat verwacht je van je onderzoek?); \textbf{Conclusie} (Wat verwacht je van van de conclusies?); \textbf{Perspectief} (Wat zegt de toekomst voor dit werk?).
	
}

%---------- Onderzoeksdomein en sleutelwoorden --------------------------------
% TODO: Vul de sleutelwoorden aan.


\Keywords{Agile, Scrum, Fixed-price, Fixed-scope, Softwaredevelopment}
\newcommand{\keywordname}{Sleutelwoorden} % Defines the keywords heading name

%---------- Titel, inhoud -----------------------------------------------------

\begin{document}
	
    \flushbottom
    %\maketitle
    \tableofcontents 
    \thispagestyle{empty} % Removes page numbering from the first page
	%------------------------------------------------------------------------------
	% Hoofdtekst
	%------------------------------------------------------------------------------
	
	\section{Inleiding}
	
	In de software wereld bestaat er een heersende cultuur die over de jaren heen opgebouwd is, namelijk de fixed-price contracts. Dit paradigma brengt echter problemen met zich mee, namelijk de kwaliteit van het product dat je oplevert. Om dit probleem op te lossen, hebben ze een methodologie, namelijk agile software development, ingevoerd. In deze casus bekijken we of het mogelijk is om zomaar over te stappen van fixed-price contracts naar het agile paradigma. Ook wordt er gekeken of je beide paradigma's kan combineren. 
	
	\section{Overzicht literatuur}
	
	% Refereren naar de literatuur kan met:
	% \autocite{BIBTEXKEY} -> (Auteur, jaartal)
	% \textcite{BIBTEXKEY} -> Auteur (jaartal)
	In het oude paradigma (fixed-price contracts), wordt er geen
	rekening gehouden met de grenzen van deze variabelen 	\autocite{Scrumology2012}. Hierbij levert men simpelweg een product tegen een bepaalde
	tijd voor een bepaalde prijs. Dit is dan ook de reden dat de kwaliteit achteruit ging bij software projecten. 
	
	Waarom is het fixed-price contract paradigma dan zo populair geweest? 
	Simpelweg omdat de klant de leverancier van de software niet betrouwt. Ze willen gewoon hun product tegen een 
	afgesproken deadline. Indien je niet aan deze eisen kan voldoen, is het makkelijk om naar de concurrentie te gaan. 
	Om de concurrentie boven te gaan, ga je rekening gaan beginnen houden met de kwaliteit van een product. We maken
	dus nu gebruik van agile. We willen absoluut kwaliteitsvolle software opleveren. 
	
	Toch is het best lastig om agile te combineren met de vastgeroeste bedrijfscultuur (fixed-price contracts). Want
	het één sluit het ander uit. Je kan geen fixed-budget, fixed-scope hanteren wanneer je kwaliteit wilt opleveren. 
	Wat wel mogelijk is, is dat één van de opgenoemde variabelen eronder lijdt. Dan kunnen we kiezen uit 
	'Money for nothing' (fixed-price met variable-scope) of 'Change for free' (variable-price met fixed-scope). 
	
    \section{Wat is agile werken met een fixed-price contract?}
    \subsection{Fixed-price contract}
     Een fixed-price-contract is een contract in project managment waar de prijs vast staat. De resources en de gespendeerde tijd staan niet vast \autocite{PMK} in theorie. In de business wereld zien we echter dat wanneer gesproken wordt over een fixed-price contract zowel scope als budget bedoeld worden \autocite{PMI2011}.
     %bron moet ik nog bijzetten --> https://project-management-knowledge.com/definitions/f/fixed-price-contract/ 
     \subsection{Agile contract}
     Agile werken beschrijft het proces waarbij iteratief en incrementeel een software product wordt opgeleverd. Men gaat te werk gebruik makende van 'sprints'. Sprints zijn tijdsintervallen (van meestal 2 weken), waarna elke sprint met de klant samen gezeten wordt om de opgeleverde en \textbf{werkende} software te bespreken en om feedback te geven. Hierdoor kan vroeger op de wijzigende verwachtingen van de klant ingespeeld worden en kunnen falende projecten nog rechtgetrokken worden. Agile werken brengt vele voordelen met zich mee maar er zijn ook een aantal nadelen aan verbonden. Voor startende teams is agile werken een echte opgave waar ze zich moeten in inwerken. Ook kost het veel tijd.
    
    \subsection{Agile met een fixed-price contract}
    Agile werken met een fixed-price contract is mogelijk maar, blijkt zeer moeilijk te zijn aangezien de fundamenten van het agile manifesto zeggen dat een project leider meer waarde zou moeten hechten aan antwoorden op verandering dan een plan te volgen (wegens de competitieve voordelen dat dit oplevert t.o.v. de concurrentie) en dit is het tegenovergestelde wat in een fixed-price/fixed-scope project verwacht wordt \autocite{PMI2011}. Hoe kan men dan tegelijkertijd zowel een wijziging verwelkomen als voorkomen?
	\section{Methodologie}
	\section{Experimenten}
    \subsection{Bedrijfsgesprek}
    Het bedrijf waar wij onze casus op zullen uitvoeren is Faction XYZ BVBA, gelegen te Antwerpen. Bij Faction gebruiken ze verschillende methodes van werken. Sommige zijn fixed price, andere zijn time \& material en nog anderen pakken ze agile aan. Faction zal ons dus een diepe kijk kunnen geven over deze verschillende methodes uitgewerkt in het bedrijfsleven en hoe deze mogelijks met elkaar kunnen overlappen en versterken of juist tegen werken.
	\section{Analyse resultaten}
	

	
	\section{Conclusie}
	

	
	%------------------------------------------------------------------------------
	% Referentielijst
	%------------------------------------------------------------------------------
	% TODO: de gerefereerde werken moeten in BibTeX-bestand ``bibliografie.bib''
	% voorkomen. Gebruik JabRef om je bibliografie bij te houden en vergeet niet
	% om compatibiliteit met Biber/BibLaTeX aan te zetten (File > Switch to
	% BibLaTeX mode)
	
	\phantomsection
	\printbibliography[heading=bibintoc]
	
\end{document}