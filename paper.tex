%==============================================================================
% Voorbeeld gebruik documentklasse hogent-article
%==============================================================================
%
% Compileren in TeXstudio:
%
% - Zorg dat Biber de bibliografie compileert (en niet Biblatex)
%   Options > Configure > Build > Default Bibliography Tool: "txs:///biber"
% - F5 om te compileren en het resultaat te bekijken.
% - Als de bibliografie niet zichtbaar is, probeer dan F5 - F8 - F5
%   Met F8 compileer je de bibliografie apart.
%
% Als je JabRef gebruikt voor het bijhouden van de bibliografie, zorg dan
% dat je in ``biblatex''-modus opslaat: File > Switch to BibLaTeX mode.

\documentclass{hogent-article}
\usepackage{lipsum} % Voor vultekst
\usepackage[backend=biber]{biblatex}
\usepackage{graphicx}
\usepackage{bmpsize}
\addbibresource{bibliografie.bib}


%------------------------------------------------------------------------------
% Metadata over het artikel
%------------------------------------------------------------------------------

%---------- Titel & auteur ----------------------------------------------------

% TODO: geef werktitel van je eigen voorstel op
\PaperTitle{De haalbaarheid van agile projects met fixed-price contracts}
% TODO: geef op welk soort artikel dit is
% Dit is typisch de opdracht en het vak waarvoor dit artikel geschreven is, bv.
\PaperType{Casus Analyse III 2019-2020}

% TODO: vul je eigen naam in als auteur, geef ook je emailadres mee!
\Authors{Indy Van Canegem\textsuperscript{1}, Mout Pessemier\textsuperscript{2}, Nante Vermeulen\textsuperscript{3}, Jef Malfliet\textsuperscript{4}} % Authors

% TODO: vul de naam van je co-promotor in.
% Als het hier gaat om een voorstel voor de bachelorproef, dan ben je hier
% verplicht de naam van je co-promotor in te vullen. Zoniet, dan kan je het
% leeg laten.
\CoPromotor{}

% Contactinfo: Geef hier de contactgegevens van elke auteur van het artikel (en
% indien van toepassing ook van de co-promotor).

\affiliation{
    \textsuperscript{1} \href{mailto:indy.vancanegem@student.hogent.be}{indy.vancanegem@student.hogent.be}}
\affiliation{
    \textsuperscript{2} \href{mailto:mout.pessemier@student.hogent.be}{mout.pessemier@student.hogent.be}}
\affiliation{
    \textsuperscript{3} \href{mailto:nante.vermeulen@student.hogent.be}{nante.vermeulen@student.hogent.be}}
\affiliation{
    \textsuperscript{4} \href{mailto:jef.malfliet@student.hogent.be}{jef.malfliet@student.hogent.be}
}

%---------- Abstract ----------------------------------------------------------

\Abstract{Werken in de IT-sector verloopt niet altijd makkelijk. Alles is relatief en lastig te begrijpen als buitenstaander. Hierdoor distantieren bedrijven zich door een simpele deadline te geven a.d.h.v fixed-price contracts, of juist nauw samen te werken met behulp van het agile manifesto. Uit de literatuurstudie blijkt dat deze onmogelijk te combineren vallen, want ze staan namelijk loodrecht op elkaar. Daarom
hebben we onderzoek gedaan aan de hand van waarheidsgetrouwe interviews binnen de IT-sector. Uit onze resultaten blijkt dat het combineren van de methodes vreemd genoeg toch mogelijk zijn in de realiteit. Ook blijkt dat een buffer opbouwen een zekere vorm van flexibiliteit kan geven, en mocht de buffer niet groot genoeg zijn, betaald de klant toch gewoon meer.
	
}

%---------- Onderzoeksdomein en sleutelwoorden --------------------------------
% TODO: Vul de sleutelwoorden aan.


\Keywords{Agile, Scrum, Fixed-price, Fixed-scope, Software development}
\newcommand{\keywordname}{Sleutelwoorden} % Defines the keywords heading name

%---------- Titel, inhoud -----------------------------------------------------

\begin{document}
	
    \flushbottom
    \maketitle
    \tableofcontents 
    \thispagestyle{empty} % Removes page numbering from the first page
	%------------------------------------------------------------------------------
	% Hoofdtekst
	%------------------------------------------------------------------------------
	
	\section{Inleiding}
	 Een fixed-price agile project is een van nature tegenstrijdige concept. De twee filosofieën die schuilgaan in de term staan loodrecht tegenover elkaar.
     
     Fixed-price contracts (FPCs) zijn lange tijd de enige manier geweest om afspraken te stellen tussen klant en leverancier. Deze contracten minimaliseren tot op zekere hoogte het risico op een gefaald product. Dit gebeurd aan de hand van een verbod op het veranderen van de 'key business constraints' ongeacht het verschuiven van scope of het toenemen van complexiteit.
     
     Haaks hierop staat de methodologie van 'agile software development'. Hierbij is flexibiliteit van hoogste belang. Doorheen het hele agile project moet er ruimte zijn om zowel de scope als de prijs aan te passen. In deze casus bespreken we of FPCs een plaats hebben binnen een agile omgeving.
	
	\section{Overzicht literatuur}
	
	% Refereren naar de literatuur kan met:
	% \autocite{BIBTEXKEY} -> (Auteur, jaartal)
	% \textcite{BIBTEXKEY} -> Auteur (jaartal)
    
	Bij FPCs wordt er geen rekening gehouden met de verandering van de price en scope \autocite{SCRUM2012}. Deze zijn vastgelegd in het contract en kunnen onder geen enkele voorwaarden worden veranderd. Dit biedt op het eerste zich een hoop zekerheid voor de klant. Deze aantrekkelijke contracten lijken weinig potentiële risico's met zich mee te dragen maar dit blijkt echter in de praktijk anders te zijn. Het project zal niet falen op vlak van prijs. En het zal ook niet falen op vlak van scope. Een paradigma als FPC brengt echter wel problemen met zich mee als het gaat over de kwaliteit van het product. Door het cementeren van de scope en de prijs laat men de leverancier slechts een keuze om bij te sturen bij onverwachte wendingen. De kwaliteit zal steeds moeten inboeten. Dit is een valkuil die op voorhand niet is op te merken voor de klant. Hun tevredenheid over de aflevertijd en de omvang van het product zal van korte duur zijn bij hun eerste gebruik. Het tekort aan kwaliteit kan gecompenseerd worden door de aflevertijd en de scope te beperken en vice versa.
    
    Ondanks dit blijven FPCs een veelgebruikte aanpak. Dit komt vaak door de onzekerheid die voortkomt uit de relatie van een klant en een leverancier van software. Men wil zijn product tegen een afgesproken deadline en voor een vooraf bepaalde som geld. Kortom, de klant wil zekerheid. De klant wil aan de hand van een contract risico's vermijden. Echter hebben onvoorziene veranderingen 
    weinig of geen ruimte binnen deze contracten. Hier biedt agile werken een oplossing.
    
    Flexibiliteit is een van de grootste troeven van het agile paradigma. Reageren op onverwachte veranderingen is bij agile zeer makkelijk. Toch is de combinatie om agile te werken met FPCs niet vanzelfsprekend. De twee methodologieën spreken elkaar namelijk tegen. Fixed-budget - fixed-scope contracts hanteren wanneer je een kwaliteitsvol agile project wil opleveren lijkt onmogelijk. Dit kan niet zonder dat één van de eerdergenoemde variabelen eronder lijdt. De twee meest bekende manieren zijn 'Money for nothing' (fixed-price met variable-scope) en 'Change for free' (variable-price met fixed-scope). 
	
    \subsection{Wat is agile werken met een fixed-price contract?}
    \subsubsection{Fixed-price contract}
     Een FPC is een contract in project managment waar de prijs vast staat. De resources en de gespendeerde tijd staan in theorie niet vast \autocite{PMK}. In de business wereld zien we echter dat wanneer gesproken wordt over een PFC zowel scope als budget bedoeld wordt \autocite{PMI2011}.
     %bron moet ik nog bijzetten --> https://project-management-knowledge.com/definitions/f/fixed-price-contract/ 
     \subsubsection{Agile contract}
     Agile werken beschrijft het paradigma waarbij iteratief en incrementeel een software product wordt opgeleverd. Men gaat te werk aan de hand van 'sprints'. Sprints zijn tijdsintervallen (van meestal 2 weken), waarna elke sprint met de klant samen gezeten wordt om de opgeleverde en werkende software te bespreken en om feedback te geven. Hierdoor kan vroeger op nieuwe verwachtingen van de klant ingespeeld worden en kunnen falende projecten nog rechtgetrokken worden. Agile werken brengt veel voordelen met zich mee maar er zijn ook een aantal nadelen aan verbonden. Voor startende teams is agile werken een echte opgave waar ze zich moeten inwerkenen en kost het daarnaast veel tijd.
    
    \subsection{Agile met een fixed-price contract}
    Agile werken met een FPC is mogelijk, maar blijkt zeer moeilijk te zijn aangezien de fundamenten van het agile manifesto zeggen dat een project leider meer waarde zou moeten hechten aan antwoorden op verandering dan een plan te volgen (wegens de competitieve voordelen dat dit oplevert t.o.v. de concurrentie) en dit is het tegenovergestelde wat in een fixed-price/fixed-scope project verwacht wordt \autocite{PMI2011}. Hoe kan men dan tegelijkertijd zowel een wijziging verwelkomen als voorkomen?
	
	\section{Methodologie}
	We hebben gebruik gemaakt van verschillende elasticiteitstechnieken om het interview af te nemen. Het interview met Faktion XYZ was ter plaatse waarbij Jef Malfliet en Mout Pessemier beurtelings de vragen hebben gesteld. Het interview met Izit was telefonisch afgelegd. Hierbij heeft Jef Malfliet het voorwoord genomen.
	
	Hierbij zijn we gestart met het uitleggen van de casus. Waarom voeren wij dit gesprek, wat wensen wij te weten te komen, hoe zit de realiteit er uit ten opzichte van wat we in de literatuurstudie te weten gekomen zijn? Nadien hebben we onze voorbereide vragen gesteld en ingespeeld op het antwoord van de geïnterviewde. Om af te sluiten zijn we snel nog eens over alle puntjes gegaan om er zeker van te zijn dat we het goed begrepen hebben en nadien hebben we onze gesprekspartner bedankt.
	
	\section{Experimenten}
	\subsection{Vragen}
	\begin{itemize}
		\item Werken jullie met fixed-price contracts?
		\item Werken jullie agile? Zo ja, hoe en op welke manieren?
		\item Hoe werkt zo'n contract?
		\item Waarom hanteren jullie zo’n werkwijze?
		\item Wie heeft zo'n contracten al aangeboden?
		\item Welke van de 3 parameters (scope, tijd en budget) staan vast en welke kunnen gewijzigd worden of werken jullie op een andere manier zoals bijvoorbeeld door het inbouwen van buffers?
		\item Hoe flexibel is werken binnen een fixed-price contract?
		\item Als er toch een van de vastgelegde parameters moet wijzigen, hoe pakken jullie die wijziging dan aan?
		\item Geven fixed-price contracts meer business kansen?
		\item Binnen het development team, werkt een fixed-price contract goed of is dit lastiger dan andere projecten?
		\item Agile werken zorgt ervoor dat men kwaliteitsvolle projecten aflevert. Hoe combineren jullie agile werken met fixed-price contracts?
		\item Is werken op deze manier een obstakel voor jullie of geven jullie de voorkeur aan deze aanpak?
		\item Zijn er nadelen aan het werken binnen een fixed-price contract die agile niet heeft en zijn er omgekeerd voordelen aan een fixed-price contract die agile niet heeft?
	\end{itemize}
		
    \subsection{De bedrijven}
    Het eerste bedrijf waar wij een interview uitgevoerd hebben was Faktion XYZ BVBA, gelegen te Antwerpen. Bij Faction gebruiken ze verschillende methodes van werken. Sommige zijn fixed price, time \& material en nog anderen pakken ze agile aan. Faction zal ons dus een diepe kijk kunnen geven over hoe deze verschillende methodes toegepast worden in het bedrijfsleven.
    \linebreak
    \linebreak
    Het tweede gesprek hebben we telefonisch uitgevoerd met Izit BVBA te Zele. Ook hier worden verschillende werkwijzen gebruikt en gecombineerd waaronder werken met FPCs en agile.
    
    \subsection{Gesprek met Faktion XYZ}
    Bij Faktion hebben we gesproken met Laurens Lavaert, de lead developer van de start-up. Hieruit hebben we veel geleerd. Zo zijn we te weten gekomen dat het verschil in werken tussen FPCs geen verschil uitmaakt voor het development team. Ze werken altijd met deadlines (via milestones) of het nu agile of fixed-price is. Deze milestones gaan als volgt te werk: Tegen datum X wordt een nieuw product verwacht. Als dit een klein project is, is het hele project slechts één milestone. Grote projecten daarentegen worden opgedeeld in meerdere milestones waarbij elke milestone een werkend product oplevert. De eerste milestone levert natuurlijk het minimal viable product (mvp) op.
    
    Het opzetten van zo'n proces gaat als volgt te werk: onderdeel X van bedrijf Y wil product Z. Zij stappen naar Faktion om een (eventueel) FPCs vast te leggen. Dan wordt er onderhandeld met de klant, wordt er een functionele analyse toegepast en de functionele requirements vastgelegd. Deze requirements worden voorgelegd aan het development team. Op basis hiervan wordt dan ingeschat hoeveel tijd en geld dit zal kosten. Hier wordt dan ook al een buffer ingebouwd. Deze projecten worden daarna op een KANBAN manier afgewerkt en opgeleverd. Mocht er toch iets fout lopen en de voorziene tijd en budget wordt overschreven, dan wordt het probleem eerst voorgelegd aan de project manager. Die gaat de klant contacteren om het probleem te melden samen om vervolgens een oplossing te zoeken. Meestal resulteert dit in het bedrijf dat bijbetaald. Eens het project af is wordt het niet onderhouden tenzij dit deel was van het contract, of de klant nieuwe features wil.
    
    De bedrijven die eerder een fixed-price aangaan zijn grote bedrijven die zeer gestructureerd te werk gaan zoals onder andere Winston en Proximus.
    
    De beide werkflows zijn dus zeker combineerbaar volgens Faktion XYZ.
    
    \subsection{Gesprek met IZIT}
    Via een telefonisch gesprek met de zaakvoerder, Ignace De Coster, kregen wij inzage in hoe men bij IZIT BVBA werkt met FPCs. En bijgevolg ook hoe zij dit combineren met agile werken. Uit het gesprek volgde dat hoewel IZIT af en toe met FPCs werkt dit uitzonderlijk met grotere bedrijven is. Dit komt doordat de IT-sector zeer weinig met dit soort contracten werkt volgens IZIT. Het tot stand komen van een FPC verloopt bij hun in meerdere fasen.
    
    Er zal eerst een commercieel gesprek met de klant gehouden worden. Hieruit zal een functionele analyse vloeien die onder andere schetsen van het product, een klassendiagram en een databankdiagram zal bevatten. Deze vergemakkelijkt het opstellen van een contract. Het is immens moeilijk om een scope, een kostprijs en een afleverperiode vast te leggen aan de hand van het abstracte idee voor een product. Deze analyse zal deel uitmaken van het FPC dat in overleg zal worden opgesteld. Hierin zal zich ook een SLA (Service Level Agreement) bevinden.
    
    Als dit basiscontract overeen gekomen is zal dit niet meer worden veranderd. Wel zullen er nog addenda kunnen worden toegevoegd om toch flexibiliteit toe te staan. Op deze manier kunnen de opgelegde limieten bij een FPC verbogen worden. Bij het basiscontract wordt ook steeds een buffer ingecalculeerd voor de opleverdatum. IZIT werkt niet met een precieze datum maar met een opleverperiode. Op deze manier is er een bepaalde speling die kan worden gebruikt. Deze aanpak wordt ook gehandhaafd als er doorheen het project geen aanpassingen voorkomen.
    
    FPCs zijn volgens IZIT geen meerwaarde voor klanten die KMO zijn. Dit komt voornamelijk omdat de voortgang van een project binnen IT veel kan verschillen van periode tot periode. Ook weet de klant dat bij FPCs er meer zal worden betaald dan bij een alternatief contract. Dit komt met het voordeel dat men wel zal kunnen budgetteren aangezien de prijs niet zal veranderen. Voor grotere bedrijven is het aanbieden van FPCs voor IZIT wel een kleine voorsprong op de competitie.
    
    De developers werken agile aan de hand van het scrum framework. Men werkt steeds in sprints van 2 weken. Hierbij worden twee parallelle sprints gehouden. Een develop sprint en een debug sprint. Op deze manier worden de oplossingen voor bugs en problemen gescheiden gehouden en bied dit een goed overzicht. Er wordt soms in sprints van een maand gewerkt als de aanpassing aan het product of de toevoeging van een feature te groot is om in 2 weken af te werken. Een sprint zal nooit langer dan een maand duren. 
    
     De combinatie van agile werken binnen een FPC levert voor IZIT geen problemen op. Er wordt geen verschil gemerkt tussen werken onder een FPC of een ander contract. De verklaring die Mr de Coster hiervoor gaf was het feit dat bij andere projecten men ook geen carte blanche krijgt. Men zal steeds attent moeten zijn bij het ontwikkelen van een project, ongeacht het contract. Af en toe zorgen de vaste limieten bij een FCP  voor wat meer stress en onrust. Maar over het algemeen zullen er geen problemen optreden bij het agile werken onder FPC's.
    
	
	\section{Conclusie}
	Uit beide gesprekken blijkt dat agile werken wel degelijk te combineren valt met FPCs mits enige nuances wat de literatuur studie tegenspreekt. Beide bedrijven bouwen een soort buffer in voor elk FPC. Deze buffer kan bestaan uit tijd of geld. Doorheen het process wordt er dan op een of andere manier (KANBAN, scrum of nog een andere manier) agile gewerkt om zo proberen tijdig het contract af te werken. Als dit toch niet lukt, dan wordt er in overleg met de klant besproken wat er nog gedaan/kan gedaan worden om het project zo succesvol mogelijk te houden. Meestal komt dit neer op het feit dat de klant bijbetaald om het project af te werken. Voor het development team dat werkt aan het contract maakt het niet uit of ze nu een fixed-price opdracht krijgen of niet. Dezelfde manier van werken wordt gehanteerd. Ook blijkt dat vooral grote bedrijven en overheidsbedrijven FPCs aangaan.

	
	%------------------------------------------------------------------------------
	% Referentielijst
	%------------------------------------------------------------------------------
	% TODO: de gerefereerde werken moeten in BibTeX-bestand ``bibliografie.bib''
	% voorkomen. Gebruik JabRef om je bibliografie bij te houden en vergeet niet
	% om compatibiliteit met Biber/BibLaTeX aan te zetten (File > Switch to
	% BibLaTeX mode)
	
	\phantomsection
	\printbibliography[heading=bibintoc]
	
\end{document}