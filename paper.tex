%==============================================================================
% Voorbeeld gebruik documentklasse hogent-article
%==============================================================================
%
% Compileren in TeXstudio:
%
% - Zorg dat Biber de bibliografie compileert (en niet Biblatex)
%   Options > Configure > Build > Default Bibliography Tool: "txs:///biber"
% - F5 om te compileren en het resultaat te bekijken.
% - Als de bibliografie niet zichtbaar is, probeer dan F5 - F8 - F5
%   Met F8 compileer je de bibliografie apart.
%
% Als je JabRef gebruikt voor het bijhouden van de bibliografie, zorg dan
% dat je in ``biblatex''-modus opslaat: File > Switch to BibLaTeX mode.

\documentclass{hogent-article}
\usepackage{lipsum} % Voor vultekst
\usepackage[backend=biber]{biblatex}
\usepackage{graphicx}
\usepackage{bmpsize}
\addbibresource{bibliografie.bib}


%------------------------------------------------------------------------------
% Metadata over het artikel
%------------------------------------------------------------------------------

%---------- Titel & auteur ----------------------------------------------------

% TODO: geef werktitel van je eigen voorstel op
\PaperTitle{De haalbaarheid van agile projects met fixed-price contracts}
% TODO: geef op welk soort artikel dit is
% Dit is typisch de opdracht en het vak waarvoor dit artikel geschreven is, bv.
\PaperType{Casus Analyse III 2019-2020}

% TODO: vul je eigen naam in als auteur, geef ook je emailadres mee!
\Authors{Indy Van Canegem\textsuperscript{1}, Mout Pessemier\textsuperscript{2}, Nante Vermeulen\textsuperscript{3}, Jef Malfliet\textsuperscript{4}} % Authors

% TODO: vul de naam van je co-promotor in.
% Als het hier gaat om een voorstel voor de bachelorproef, dan ben je hier
% verplicht de naam van je co-promotor in te vullen. Zoniet, dan kan je het
% leeg laten.
\CoPromotor{}

% Contactinfo: Geef hier de contactgegevens van elke auteur van het artikel (en
% indien van toepassing ook van de co-promotor).

\affiliation{
    \textsuperscript{1} \href{mailto:indy.vancanegem@student.hogent.be}{indy.vancanegem@student.hogent.be}}
\affiliation{
    \textsuperscript{2} \href{mailto:mout.pessemier@student.hogent.be}{mout.pessemier@student.hogent.be}}
\affiliation{
    \textsuperscript{3} \href{mailto:nante.vermeulen@student.hogent.be}{nante.vermeulen@student.hogent.be}}
\affiliation{
    \textsuperscript{4} \href{mailto:jef.malfliet@student.hogent.be}{jef.malfliet@student.hogent.be}
}

%---------- Abstract ----------------------------------------------------------

\Abstract{Hier schrijf je de samenvatting van je artikel, als een doorlopende tekst van één paragraaf. Wat hier zeker in moet vermeld worden: \textbf{Context} (Waarom is dit werk belangrijk?); \textbf{Nood} (Waarom moet dit onderzocht worden?); \textbf{Taak} (Wat ga je (ongeveer) doen?); \textbf{Object} (Wat staat in dit document geschreven?); \textbf{Resultaat} (Wat verwacht je van je onderzoek?); \textbf{Conclusie} (Wat verwacht je van van de conclusies?); \textbf{Perspectief} (Wat zegt de toekomst voor dit werk?).
	
}

%---------- Onderzoeksdomein en sleutelwoorden --------------------------------
% TODO: Vul de sleutelwoorden aan.


\Keywords{Agile, Scrum, Fixed-price, Fixed-scope, Softwaredevelopment}
\newcommand{\keywordname}{Sleutelwoorden} % Defines the keywords heading name

%---------- Titel, inhoud -----------------------------------------------------

\begin{document}
	
    \flushbottom
    \maketitle
    \tableofcontents 
    \thispagestyle{empty} % Removes page numbering from the first page
	%------------------------------------------------------------------------------
	% Hoofdtekst
	%------------------------------------------------------------------------------
	
	\section{Inleiding}
	 Een fixed-Price agile project is een van nature tegenstrijdige concept. De twee filosofieën die schuilgaan in de term staan loodrecht tegenover elkaar.
     
     Fixed-price contracts (FPCs) zijn lange tijd de enige manier geweest om afspraken te stellen tussen klant en leverancier. Deze contracten minimaliseren tot op zekere hoogte het risico op een gefaald product. Dit gebeurd aan de hand van een verbod op het veranderen van de 'key business constraints' ongeacht het verschuiven van scope of het toenemen van complexiteit.
     
     Haaks hierop staat de methodologie van 'agile software development'. Hierbij is flexibiliteit van hoogste belang. Doorheen het hele agile project moet er ruimte zijn om zowel de scope als de prijs aan te passen. In deze casus bespreken we of FPCs een plaats hebben binnen een agile omgeving.
	
	\section{Overzicht literatuur}
	
	% Refereren naar de literatuur kan met:
	% \autocite{BIBTEXKEY} -> (Auteur, jaartal)
	% \textcite{BIBTEXKEY} -> Auteur (jaartal)
    
	De fixed-price contracts wordt er geen rekening gehouden met de verandering van deze price en scope \autocite{Scrumology2012}. Deze zijn vastgelegd in het contract en kunnen onder geen enkele voorwaarden worden veranderd. Dit bied op het eerste zich een hoop zekerheid voor de klant. Deze aantrekkelijke contracten lijken weinig potentiële risico's met zich mee te dragen maar dit blijkt echter in de praktijk anders te zijn. Het project zal niet falen op vlak van prijs. En het zal ook niet falen op vlak van scope. Een paradigma als FPC brengt echter wel problemen met zich mee als het gaat over de kwaliteit van het product. Door het cementeren van de scope en de prijs laat men de leverancier slechts een keuze om bij te sturen bij onverwachte wendingen. De kwaliteit zal steeds moeten inboeten. Dit is een valkuil die op voorhand niet is op te merken voor de klant. Hun tevredenheid over de aflevertijd en de omvang van het product zal van korte duur zijn bij hun eerste gebruik. Het tekort aan kwaliteit heeft onvoorziene veranderingen in scope of aflevertijd moeten compenseren.
    
    Ondanks dit blijfen FPCs een veelgebruikte aanpak. Dit komt vaak door de onzekerheid die voortkomt uit de relatie van een klant en een leverancier van software. Men wil zijn product tegen een afgesproken deadline. En voor een vooraf bepaalde som geld. Kortom, de klant wilt zekerheid. Men wil op voorhand de te voorziene risico's vermijden. In deze taak slaagt een FPC compleet. Te dure activiteiten, halve of foute implementaties, te lange aflevertijd, enz. Deze problemen worden omzeild of opgelost met behulp van een dergelijk contract. Het grote probleem met deze aanpak is net datgene dat het zo aantrekkelijk maakt. De onvoorziene veranderingen hebben weinig manieren om te worden opgelost. Hier bied agile werken hulp.
    
    Flexibiliteit is een van de grootste troeven van het agile paradigma. Reageren op onverwachte veranderingen is bij agile zeer gemakkelijk. Toch is de combinatie agile te werken met FPCs niet vanzelfsprekend. De twee methodologieën sluiten verre van naadloos aan op elkaar. Een fixed-budget, fixed-scope contract hanteren wanneer je een kwaliteitsvol agile project wil opleveren lijkt onmogelijk. Dit kan niet zonder dat één van de eerdergenoemde variabelen eronder lijdt. De twee meest bekende manieren om dit op een gestructureerde manier te doen kiezen zijn 'Money for nothing' (fixed-price met variable-scope) en 'Change for free' (variable-price met fixed-scope). 
	
    \subsection{Wat is agile werken met een fixed-price contract?}
    \subsubsection{Fixed-price contract}
     Een fixed-price-contract is een contract in project managment waar de prijs vast staat. De resources en de gespendeerde tijd staan niet vast \autocite{PMK} in theorie. In de business wereld zien we echter dat wanneer gesproken wordt over een fixed-price contract zowel scope als budget bedoeld worden \autocite{PMI2011}.
     %bron moet ik nog bijzetten --> https://project-management-knowledge.com/definitions/f/fixed-price-contract/ 
     \subsubsection{Agile contract}
     Agile werken beschrijft het proces waarbij iteratief en incrementeel een software product wordt opgeleverd. Men gaat te werk gebruik makende van 'sprints'. Sprints zijn tijdsintervallen (van meestal 2 weken), waarna elke sprint met de klant samen gezeten wordt om de opgeleverde en werkende software te bespreken en om feedback te geven. Hierdoor kan vroeger op de wijzigende verwachtingen van de klant ingespeeld worden en kunnen falende projecten nog rechtgetrokken worden. Agile werken brengt vele voordelen met zich mee maar er zijn ook een aantal nadelen aan verbonden. Voor startende teams is agile werken een echte opgave waar ze zich moeten in inwerken. Ook kost het veel tijd.
    
    \subsection{Agile met een fixed-price contract}
    Agile werken met een fixed-price contract is mogelijk maar, blijkt zeer moeilijk te zijn aangezien de fundamenten van het agile manifesto zeggen dat een project leider meer waarde zou moeten hechten aan antwoorden op verandering dan een plan te volgen (wegens de competitieve voordelen dat dit oplevert t.o.v. de concurrentie) en dit is het tegenovergestelde wat in een fixed-price/fixed-scope project verwacht wordt \autocite{PMI2011}. Hoe kan men dan tegelijkertijd zowel een wijziging verwelkomen als voorkomen?
	\section{Methodologie}
	We hebben gebruik gemaakt van verschillende elasticiteitstechnieken om het interview af te nemen.
	\section{Experimenten}
    \subsection{De Bedrijven}
    Het eerste bedrijf waar wij onze casus op zullen uitvoeren is Faktion XYZ BVBA, gelegen te Antwerpen. Bij Faction gebruiken ze verschillende methodes van werken. Sommige zijn fixed price, andere zijn time \& material en nog anderen pakken ze agile aan. Faction zal ons dus een diepe kijk kunnen geven over deze verschillende methodes uitgewerkt in het bedrijfsleven en hoe deze mogelijks met elkaar kunnen overlappen en versterken of juist tegen werken.
    \linebreak
    \linebreak
    Het tweede gesprek hebben we telefonisch uitgevoerd met Izit BVBA te Zele. Ook hier worden verschillende werkwijzen gebruikt en gecombineerd waaronder werken via fixed-price contracts en agile werken.
    
    \subsection{Gesprek met Faktion XYZ}
    Bij Faktion hebben we gesproken met Laurens Laveart, de lead developer van de startup. Hieruit hebben we veel geleerd. Zo zijn we te weten gekomen dat het verschil in werken tussen fixed-price contracts geen verschil uitmaakt voor het development team. Ze werken altijd met deadlines (via milestones) of het nu agile of fixed-price is. Deze milestones gaan als volgt te werk: Tegen datum X wordt een nieuw product verwacht. Als dit een klein project is, is het hele project slechts één milestone. Grote projecten daarentegen worden opgedeeld in meerdere milestones waarbij elke milestone een werkend product oplevert. De eerste milestone levert natuurlijk het minimal viable product (mvp) op.
    
    Het opzetten van zo'n proces gaat als volgt tewerk: onderdeel X van bedrijf Y wil product Z. Zij stappen naar Faktion om een (eventueel) fixed-price contract vast te leggen. Dan wordt er onderhandeld met de klant, wordt er een functionele analyse toegepast en de functionele requirements vastgelegd. Deze requirements worden voorgelegd aan het development team. Op basis hiervan word dan ingeschat hoeveel tijd en geld dit zal kosten. Hier worden dan ook al buffer ingebouwd. Deze projecten worden daarna op een KANBAN manier afgewerkt en opgeleverd. Moest er toch iets fout lopen en er zou over tijd of budget moeten gegeaan worden, dan wordt het probleem eerst voorgelegd aan de project manager. Die gaat de klant contacteren om het probleem te meldenen samen een oplossing te zoeken. Meestal resulteert dit in het bedrijf dat bijbetaald. Eens het project af is wordt het niet onderhouden tenzij dit deel was van het contract of als de klant nieuwe features wilt.
    
    De bedrijven die eerder een fixed-price aangaan zijn grote bedrijven die zeer gestructureerd te werk gaan zoals onder andere Winston en Proximus.
    
    De beide werkflows zijn dus zeker combineerbaar.
    
    \subsection{Gesprek met Izit}
	\section{Analyse resultaten}
	

	
	\section{Conclusie}
	

	
	%------------------------------------------------------------------------------
	% Referentielijst
	%------------------------------------------------------------------------------
	% TODO: de gerefereerde werken moeten in BibTeX-bestand ``bibliografie.bib''
	% voorkomen. Gebruik JabRef om je bibliografie bij te houden en vergeet niet
	% om compatibiliteit met Biber/BibLaTeX aan te zetten (File > Switch to
	% BibLaTeX mode)
	
	\phantomsection
	\printbibliography[heading=bibintoc]
	
\end{document}